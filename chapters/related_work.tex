\chapter{Related Work}
This chapter provides a comprehensive overview of existing literature relevant to Elasticity and AIF within distributed and edge computing contexts. The discussion is structured around three thematic sections: 

\section{Elasticity for stream processing under SLO constraints}
In \cite{sedlak_towards_2025}, Sedlak et al. present a control architecture where services can adapt along multiple dimensions of elasticity, i.e., Resource- and Quality Elasticity. The system dynaimcally adjusts configuration parameters such as the number of allocated CPU cores (\textit{cpu}) and the resolution of the video stream (\textit{pixels}). These two controllable parameters, along with the frame rate (\textit{fps}) of the video stream, are used as SLO targets for the stream processing services. Actions are selected based on a combination of Deep Reinforcement Learning and a reward function, based on SLO fulfillment. Rather than relying solely on vertical autoscaling, their framework uses elasticity as a first-class mechanism to operate under fixed-resource budgets typical of edge nodes. This work also introduces the idea of causally linking low-level service parameters with high-level application goals.

\section{Active Inference as a unified approach for edge adaptation}
In \cite{sedlak_active_2024}, Sedlak et al. present a design study for applying Active Inference (AIF) to edge computing systems with a focus on maintaining Service Level Objectives (SLOs) in resource-constrained environments. They propose a self-evidencing agent that operates in an action-perception cycle, using a generative model to predict system behavior, compare outcomes with expectations, and update beliefs to minimize free energy. The agent’s preferences are encoded directly as SLOs, guiding both its predictions and actions. The framework explicitly links system metrics—such as batch delay and utilization—to causal structures, enabling interpretability and continuous adaptation without requiring pretraining. This approach distinguishes itself from traditional ML systems by integrating causal reasoning, homeostasis, and epistemic exploration into a unified control loop. This design study provides the blueprint for implementing AIF-based control in edge computing scenarios.

Complementing this theoretical design, \cite{sedlak_adaptive_2024} presents a fully implemented and empirically evaluated AIF agent for adaptive stream processing on edge devices. Their system dynamically adjusts stream quality parameters—specifically frame rate (fps), resolution (pixel), and processing mode—based on the agent’s generative model and its beliefs about system behavior. SLOs such as latency, energy consumption, and detection accuracy are encoded as preferred observations (priors), guiding the agent’s policy selection through the minimization of EFE. The action-perception cycle allows the agent to iteratively update its beliefs and reconfigure services in response to observed deviations, thereby ensuring consistent SLO fulfillment across heterogeneous devices and workloads. This work directly informs the methodology of this thesis, which adopts a comparable architectural approach and operationalizes SLOs in a similar way to enable self-adaptive control.



