\chapter{Conclusion}
\label{chap:conclusion}

This thesis investigated the application of the Active Inference (AIF) framework for upholding Service Level Objectives (SLOs) in distributed edge computing scenarios. A self-adaptive stream processing pipeline was developed, leveraging AIF to regulate elasticity along three dimensions: frame rate, resolution, and model quality. The objective was to maintain high Quality of Experience (QoE) while operating under resource constraints.

The implemented prototype integrates an AIF agent at the producer node. By minimizing expected free energy, the agent continuously selects actions that reduce the risk of future SLO violations. Online learning through parameter updates enables the agent to adapt its generative model during runtime, facilitating behavior adjustment without requiring offline retraining.

Evaluation showed that the AIF-based elasticity mechanism achieves comparable or superior global SLO compliance relative to a heuristic baseline across varying conditions.While the heuristic performs better in maximizing stream quality under high load, the AIF agent consistently prioritizes SLO fulfillment, even at the cost of reduced output quality.

The Active Inference agent demonstrates clear advantages in stable and moderately dynamic environments, where it maintains high SLO compliance through deliberate and consistent adaptation. Its preference for equilibrium leads to minimal parameter oscillation and robust long-term performance. However, in highly volatile scenarios with abrupt workload changes, the agent exhibits delayed adaptation due to its reliance on prior beliefs. This highlights a key limitation of the current approach: a lack of fast re-adaptation mechanisms.

While the evaluation focused on video inference, the presented approach generalizes to other data modalities, such as LIDAR, audio streams, and sensor telemetry. 

In summary, this thesis contributes a principled and extensible method for adaptive control in edge computing. By unifying probabilistic reasoning, preference-driven adaptation, and online learning, the proposed system provides a novel foundation for resilient, self-organizing stream processing under uncertainty.
