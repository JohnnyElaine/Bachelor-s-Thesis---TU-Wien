% Copyright (C) 2014-2024 by Thomas Auzinger <thomas@auzinger.name>

\documentclass[draft,final]{vutinfth} % Remove option 'final' to obtain debug information.

% Define convenience functions to use the author name and the thesis title in the PDF document properties.
\newcommand{\authorname}{Elias Huhsovitz} % The author name without titles.
\newcommand{\thesistitle}{Elastic Self-Adaptive Edge Pipelines} % The title of the thesis. The English version should be used, if it exists.

% Create the XMP metadata file for the creation of PDF/A compatible documents.
\begin{filecontents*}[overwrite]{\jobname.xmpdata}
\Author{\authorname}                                    % The author's name in the document properties.
\Title{\thesistitle}                                    % The document's title in the document properties.
\Language{de-AT}                                        % The document's language in the document properties. Select 'en-US', 'en-GB', or 'de-AT'.
\Keywords{a\sep list\sep of\sep keywords}               % The document's keywords in the document properties (separated by '\sep ').
\Publisher{TU Wien}                                     % The document's publisher in the document properties.
\Subject{Thesis}                                        % The document's subject in the document properties.
\end{filecontents*}

% Load packages to allow in- and output of non-ASCII characters.
\usepackage{lmodern}        % Use an extension of the original Computer Modern font to minimize the use of bitmapped letters.
\usepackage[T1]{fontenc}    % Determines font encoding of the output. Font packages have to be included before this line.
\usepackage[utf8]{inputenc} % Determines encoding of the input. All input files have to use UTF8 encoding.

% Extended LaTeX functionality is enables by including packages with \usepackage{...}.
\usepackage{amsmath}    % Extended typesetting of mathematical expression.
\usepackage{amssymb}    % Provides a multitude of mathematical symbols.
\usepackage{mathtools}  % Further extensions of mathematical typesetting.
\usepackage{microtype}  % Small-scale typographic enhancements.
\usepackage[inline]{enumitem} % User control over the layout of lists (itemize, enumerate, description).
\usepackage{multirow}   % Allows table elements to span several rows.
\usepackage{booktabs}   % Improves the typesetting of tables.
\usepackage{subcaption} % Allows the use of subfigures and enables their referencing.
\usepackage[ruled,linesnumbered,algochapter]{algorithm2e} % Enables the writing of pseudo code.
\usepackage[dvipsnames,table]{xcolor} % Allows the definition and use of colors. This package has to be included before tikz.
\usepackage{nag}        % Issues warnings when best practices in writing LaTeX documents are violated.
\usepackage{todonotes}  % Provides tooltip-like todo notes.
\usepackage{morewrites} % Increases the number of external files that can be used.
\usepackage[a-2b,mathxmp]{pdfx}      % Enables PDF/A compliance. Loads the package hyperref and has to be included second to last.
\usepackage[acronym,toc]{glossaries} % Enables the generation of glossaries and lists of acronyms. This package has to be included last.

% packages added by elias huhsovitz
\usepackage{float}

% used for tikz vector drawing
\usepackage{pgfplots}
\pgfplotsset{compat=1.18}

% Set PDF document properties
\hypersetup{
    pdfpagelayout   = TwoPageRight,           % How the document is shown in PDF viewers (optional).
    linkbordercolor = {Melon},                % The color of the borders of boxes around hyperlinks (optional).
}

\setpnumwidth{2.5em}        % Avoid overfull hboxes in the table of contents (see memoir manual).
\setsecnumdepth{subsection} % Enumerate subsections.

\nonzeroparskip             % Create space between paragraphs (optional).
\setlength{\parindent}{0pt} % Remove paragraph indentation (optional).

\makeindex      % Use an optional index.
\makeglossaries % Use an optional glossary.
%\glstocfalse   % Remove the glossaries from the table of contents.

% Set persons with 4 arguments:
%  {title before name}{name}{title after name}{gender}
%  where both titles are optional (i.e. can be given as empty brackets {}).
\setauthor{}{\authorname}{}{male}
% \setadvisor{Univ.Prof. Dr.}{Schahram Dustdar}{}{male}
\setadvisor{Proj.Ass. Dr.}{Boris Sedlak}{}{male}

% For bachelor and master theses:
% \setfirstassistant{Proj.Ass. Dr.}{Boris Sedlak}{}{male}
%\setsecondassistant{Pretitle}{Forename Surname}{Posttitle}{male}
%\setthirdassistant{Pretitle}{Forename Surname}{Posttitle}{male}

% For dissertations:
\setfirstreviewer{Pretitle}{Forename Surname}{Posttitle}{male}
\setsecondreviewer{Pretitle}{Forename Surname}{Posttitle}{male}

% For dissertations at the PhD School and optionally for dissertations:
\setsecondadvisor{Pretitle}{Forename Surname}{Posttitle}{male} % Comment to remove.

% Required data.
\setregnumber{12121272}
\setdate{01}{06}{2025} % Set date with 3 arguments: {day}{month}{year}.
\settitle{\thesistitle}{Active Inference for Elastic Control of Edge-Based Data Streams} % Sets English and German version of the title (both can be English or German). If your title contains commas, enclose it with additional curvy brackets (i.e., {{your title}}) or define it as a macro as done with \thesistitle.
%\setsubtitle{} % Sets English and German version of the subtitle (both can be English or German).

% Select the thesis type: bachelor / master / doctor.
% Bachelor:
\setthesis{bachelor}
%
% Master:
%\setthesis{master}
%\setmasterdegree{dipl.} % dipl. / rer.nat. / rer.soc.oec. / master
%
% Doctor:
%\setthesis{doctor}
%\setdoctordegree{rer.soc.oec.}% rer.nat. / techn. / rer.soc.oec.

% For bachelor and master:
\setcurriculum{Software \& Information Engineering}{Software \& Information Engineering} % Sets the English and German name of the curriculum.

% Optional reviewer data:
\setfirstreviewerdata{Affiliation, Country}
\setsecondreviewerdata{Affiliation, Country}


\begin{document}

\frontmatter % Switches to roman numbering.
% The structure of the thesis has to conform to the guidelines at
%  https://informatics.tuwien.ac.at/study-services

%\addtitlepage{naustrian} % German title page.
\addtitlepage{english} % English title page.
\addstatementpage

%\begin{danksagung*}
%\todo{Ihr Text hier.}
%\end{danksagung*}

\begin{acknowledgements*}
I would like to thank my supervisor, Dr. Boris Sedlak, for his guidance and helpful feedback during the development of this thesis.

I also thank Felix Berger and Tobias Seczer for their valuable non-technical feedback, particularly regarding figure layout and document presentation.
\end{acknowledgements*}

%\begin{kurzfassung}
%\todo{Ihr Text hier.}
%\end{kurzfassung}

\begin{abstract}
This thesis explores how the Active Inference (AIF) framework can be employed to uphold Service Level Objectives (SLOs) in resource-constrained edge computing environments. A novel distributed stream processing pipeline is designed and implemented, enabling real-time video inference across parallel worker nodes. Elasticity is expressed through dynamic adjustment of three stream quality parameters: frame rate, resolution, and model complexity.

An AIF agent deployed at the producer node continuously monitors the system state and minimizes expected free energy to select actions that reduce the risk of future SLO violations. Online learning is achieved through runtime updates to the agent's generative model. SLOs—such as task queue size, processing latency, and memory usage—are encoded as prior preferences, allowing the agent to balance constraint satisfaction with stream quality maximization.

The prototype is evaluated against a heuristic baseline across three scenarios: a static environment, dynamic demand variation, and fluctuating computational budget. Results show that the AIF agent outperforms the baseline in stable and moderately dynamic environments, maintaining higher global SLO satisfaction and more stable configurations. However, performance degrades under highly volatile workloads due to limited reactivity. This highlights the need for future extensions supporting faster re-adaptation and distributed control across multiple nodes.

Overall, this work presents a principled method for elastic, self-adaptive control in edge systems. By integrating probabilistic reasoning, preference-driven adaptation, and online model learning, Active Inference provides a unified control paradigm for resilient stream processing under uncertainty.
\end{abstract}


% Select the language of the thesis, e.g., english or naustrian.
\selectlanguage{english}

% Add a table of contents (toc).
\tableofcontents % Starred version, i.e., \tableofcontents*, removes the self-entry.

% Switch to arabic numbering and start the enumeration of chapters in the table of content.
\mainmatter

% link to chapters
\chapter{Introduction}
\todo{Problem Statement}
General parallel processing pipeline for edge computing cases, when the tasks are real time. For evaluation purposes, we used a video stream. For computation we used YOLOv11 inference.
Goal: uphold a certain level of QoE while ensuring usable inference results.
\chapter{Background}
This chapter introduces foundational concepts essential for understanding how Active Inference
can be effectively leveraged to uphold Service Level Objectives (SLOs) in resource-constrained
edge computing environments. The first section establishes the critical role of SLOs in edge computing scenarios. The subsequent section explores all of the key concepts regarding Active Inference and the Free Energy Principle. The final section discusses the application of Active Inference in distributed systems.

\section{Edge Computing and Service Level Objectives}
Edge computing refers to a distributed computing paradigm in which data processing occurs in
close proximity to the data source. This architectural shift reduces the latency and bandwidth
limitations inherent in centralized cloud computing \cite{deng_edge_2020}, while enabling real-time
analytics for latency-sensitive tasks such as autonomous driving, smart surveillance, and
industrial automation \cite{zhang_octopus_2023}.

Edge computing environments are characterized by pronounced heterogeneity \cite{danilenka_adaptive_2025}: devices may
range from microcontrollers and single-board computers to more capable edge servers, each
with distinct resource constraints in terms of CPU, memory, energy, bandwidth and especially GPU, which has become a significant factor due to increasing inference demand. When tasked with continuous stream
processing (e.g., real-time video analysis), such systems must dynamically manage
computational demands without the luxury of cloud-level resource elasticity.

To formalize quality expectations in such constrained environments, Service Level Objectives
(SLOs) are used \cite{casamayor_pujol_deepslos_2024}. SLOs define quantifiable thresholds on performance metrics, such as
response time, memory usage, or energy consumption targets \cite{danilenka_adaptive_2025}. SLOs serve as internal optimization goals \cite{danilenka_adaptive_2025} to guide system behavior under varying load conditions \cite{nastic_sloc_2020}.
For example, in a video inference pipeline, one
might define an SLO that memory usage must remain below 80\% of available capacity.

Crucially, SLOs can serve a dual role: they can be targets for optimization and/or constraints that the
system must not violate \cite{casamayor_pujol_deepslos_2024}, \cite{sedlak_diffusing_2024}. In edge scenarios, where stream quality competes directly with resource availability, such as video streaming tasks \cite{sedlak_adaptive_2024}, SLO-aware control is essential \cite{sedlak_slo-aware_2025}.

\section{Active Inference}
\section{Active Inference in Distributed Systems}


\chapter{Related Work}
This chapter provides an overview of key literature relevant to Elasticity and AIF within distributed and edge computing contexts. While AIF has found wide application in robotics and cognitive systems \cite{lanillos_active_2021}, its adoption in edge computing remains sparse. Notable progress in applying elasticity through AIF in resource-constrained environments emerged primarily through the work of Sedlak et al. \cite{sedlak_adaptive_2024, sedlak_equilibrium_2024, sedlak_active_2024, sedlak_slo-aware_2025, lapkovskis_benchmarking_2025, sedlak_towards_2025}, Danilenka et al. \cite{danilenka_adaptive_2025}, and to a certain extent, Levchuk et al. \cite{levchuk_active_2019} in multi-agent Internet
of Things (IoT) systems. Given the early stage of AIF in edge computing, this chapter focuses on a selected set of foundational and state-of-the-art contributions to contextualize the motivation and relevance of our approach.

\section{Elasticity for stream processing under SLO constraints}
In \cite{sedlak_towards_2025}, Sedlak et al. present a control architecture where services can adapt along multiple dimensions of elasticity, i.e., Resource- and Quality Elasticity. The system dynaimcally adjusts configuration parameters such as the number of allocated CPU cores (\textit{cpu}) and the resolution of the video stream (\textit{pixels}). These two controllable parameters, along with the frame rate (\textit{fps}) of the video stream, are used as SLO targets for the stream processing services. Actions are selected based on a combination of Deep Reinforcement Learning and a reward function, based on SLO fulfillment. Rather than relying solely on vertical autoscaling, their framework uses elasticity as a first-class mechanism to operate under fixed-resource budgets typical of edge nodes. This work also introduces the idea of causally linking low-level service parameters with high-level application goals.

\section{Active Inference as a unified approach for edge adaptation}
In \cite{sedlak_active_2024}, Sedlak et al. present a design study for applying Active Inference (AIF) to edge computing systems with a focus on maintaining Service Level Objectives (SLOs) in resource-constrained environments. They propose a self-evidencing agent that operates in an action-perception cycle, using a generative model to predict system behavior, compare outcomes with expectations, and update beliefs to minimize free energy. The agent’s preferences are encoded directly as SLOs, guiding both its predictions and actions. The framework explicitly links system metrics—such as batch delay and utilization—to causal structures, enabling interpretability and continuous adaptation without requiring pretraining. This approach distinguishes itself from traditional ML systems by integrating causal reasoning, homeostasis, and epistemic exploration into a unified control loop. This design study provides the blueprint for implementing AIF-based control in edge computing scenarios.

Complementing this theoretical design, \cite{sedlak_adaptive_2024} presents an implemented and evaluated AIF agent for adaptive stream processing on edge devices. Their system dynamically adjusts stream quality parameters—specifically frame rate (fps), resolution (pixel), and processing mode—based on the agent’s generative model and its beliefs about system behavior. SLOs such as latency, energy consumption, and detection accuracy are encoded as preferred observations (priors), guiding the agent’s policy selection through the minimization of EFE. The action-perception cycle allows the agent to iteratively update its beliefs and reconfigure services in response to observed deviations, thereby ensuring consistent SLO fulfillment across heterogeneous devices and workloads. This work directly informs the methodology of this thesis, which adopts a comparable architectural approach and operationalizes SLOs in a similar way to enable self-adaptive control.



\chapter{Methodology}
\todo{Enter your text here.}
In this chapter, the methods used are explained in detail. Firstly, 
\chapter{Implementation}
\label{chap:implementation}
This chapter presents the implementation of the proposed methodology outlined
in chapter~\ref{chap:methodology}. It demonstrates the feasibility of using Active Inference (AIF) to dynamically manage elasticity in distributed edge pipelines while upholding Service Level Objectives (SLOs). A Python-based prototype simulates a real-time video processing system and compares the AIF agent to baseline control strategies under constrained edge conditions.

The prototype of the distributed pipeline and the experimental results are available in a publicly accessible repository \footnote{The prototype of the stream processing framework is available at \href{https://github.com/JohnnyElaine/bsc_aif_parallel_pipeline}{GitHub}, accessed on July 22nd 2025.}.


The prototype implements a parallel and distributed processing pipeline inspired by Apache Flink~\cite{carbone_apache_2015}. It processes video streams using YOLOv11 inference and dynamically controls elasticity via an AIF agent deployed at the producer.

\section{System Architecture}
The architecture consists of three roles:
\begin{itemize}
    \item \textbf{Producer:} Splits the frames of the video stream into tasks, and dispatches them to workers. It also acts as the central controller of elasticity and exposes control of the \textit{stream quality parameters} to its AIF agent.
    \item \textbf{Workers:} Execute YOLOv11 inference on received tasks and sends the result to the Collector.
    \item \textbf{Collector:} Aggregates processed results into an ordered output stream.
\end{itemize}

Communication is fully asynchronous and implemented using ZeroMQ sockets~\cite{noauthor_zeromqpyzmq_nodate}. Task dispatch from the Producer to Workers uses the REQ-ROUTER pattern, forming a pull-based work distribution model, also known as a Load-Balancing Pattern. Processed results are forwarded from Workers to the Collector using the PUSH-PULL pattern. This design ensures that work is allocated based on each worker’s readiness, optimizing system responsiveness and throughput.

ZeroMQ sockets are multiplexed using event loops, allowing simultaneous listening and sending. Task payloads (NumPy arrays) are transmitted as raw binary data without any copying, by leveraging the buffer interface they implement. Task metadata is serialized using msgpack \cite{noauthor_msgpackmsgpack-python_nodate} for compact binary transport.


\section{Producer Architecture}
The producer fulfills three roles: task generation, elasticity control, and agent orchestration.

\subsection{Task Queue and Generation}
A continuous stream of video frames is segmented into tasks and temporarily stored in a FIFO queue. Each \texttt{Task} includes:
\begin{itemize}
    \item \texttt{type:} Task purpose (e.g., INFERENCE, COLLECT).
    \item \texttt{id:} Unique identifier.
    \item \texttt{stream\_key:} Identifier for multi-stream support.
    \item \texttt{data:} A video frame, as a NumPy \texttt{ndarray}.
\end{itemize}
The number of tasks generated per second is dependent on the current configuration of the \textit{fps} parameter. Tasks are served upon worker request and consumed from the queue.

\subsection{Service Level Objectives}

Table~\ref{tab:slo-table} shows 4 types of SLOs intending to guarantee the QoE. These SLOs serve as constraints that guide the elastic adaptation process of the producer, helping to maintain a balance between processing quality and system stability.

\begin{table}[h]
    \centering
    \begin{tabular}{@{}lll@{}}
        \toprule
        \textbf{Var.} & \textbf{Rel.} & \textbf{Description} \\
        \midrule
        \textit{memory usage} 
            & \( \leq \theta_\text{mem} \) 
            & container memory usage \\
            
        \textit{task queue size} 
            & \( \leq \textit{fps} \cdot \varepsilon_\text{queue} \) 
            & task queue size \\
            
        \textit{avg process time} 
            & \( \leq \frac{1}{\textit{fps}} \cdot \varepsilon_\text{global}\) 
            & average global processing time \\
            
        \textit{highest avg process time} 
            & \( \leq \frac{1}{\textit{fps}} \cdot \varepsilon_\text{worker}\) 
            & worker with the highest average processing time \\
        \bottomrule
    \end{tabular}
    \caption{Service Level Objectives (SLOs) at the producer}
    \label{tab:slo-table}
\end{table}


The \textbf{Memory Usage SLO} ensures that the container does not exceed a specified memory capacity \(\theta_\text{mem}\). This protects the system from memory saturation, which would cause severe performance degradation as tasks might be offloaded to slower storage or dropped entirely.

\paragraph{Task Queue Size SLO}
Ensures that the number of unprocessed tasks remains within a reasonable limit, based on the current \textit{fps} parameter and a tolerance \(\varepsilon_\text{queue}\). A growing queue indicates that the processing capacity of the workers is insufficient for the current workload. This SLO acts as a safeguard for maintaining real-time responsiveness. The tolerance threshold must be set low enough to avoid the accumulation of multiple seconds worth of unprocessed frames in the task queue. At the same time, it must be sufficiently permissive to prevent immediate reactions to minor fluctuations, such as transient network delays. Based on empirical evaluation, a recommended tolerance range for the queue size SLO is \( 2 \geq \varepsilon_\text{queue} \geq 4\).

\paragraph{Average Global Task Processing Time SLO} Ensures that, on average, all workers combined can keep pace with the input stream. It is computed by tracking the processing time of the most recent \(n\) completed tasks using a moving average window. These statistics are reported by each worker during regular task requests. The goal is to ensure that the cumulative system throughput is sufficient to maintain real-time responsiveness. \(\varepsilon_\text{global}\) denotes a tolerance factor. A recommended value is \(\varepsilon_\text{global} = 1\), meaning the average time to process a single task must not exceed the current task generation time interval. Violations of this SLO indicate that the aggregate compute capacity of the system is insufficient to sustain the current input rate. This SLO serves as a holistic throughput constraint, ensuring that the pipeline as a whole can process frames as quickly as they arrive.

Violations of this SLO indicate that the aggregate compute capacity of the system is insufficient to sustain the current input rate. This SLO serves as a holistic throughput constraint, ensuring that the pipeline as a whole can process frames as quickly as they arrive.

\paragraph{Highest Per-Worker Average Task Processing Time SLO} Enforces an upper bound on the slowest worker’s performance. It ensures that no individual worker lags so far behind that its results become irrelevant to the output stream. This is particularly important in distributed pipelines, where late-arriving results may be discarded if the collector has already progressed past the corresponding task ID.

Here, \(\varepsilon_\text{worker} \geq 1\) provides additional tolerance compared to the global SLO. A typical setting is \(\varepsilon_\text{worker} = 4\), allowing slower nodes to process a task in up to 4 times the time budget permitted by the frame rate. By ensuring that even the slowest node performs within tolerable bounds, the system can maintain temporal coherence and avoid dropping results from late workers.

\subsection{SLO State Discretization}
To enable discrete decision-making within the Active Inference controller, the continuous \textit{SLO value} is transformed into one of three discrete SLO states:

\begin{table}[h]
    \centering
    \begin{tabular}{@{}lll@{}}
        \toprule
        \textbf{SLO State} & \textbf{SLO Value} & \textbf{Description} \\
        \midrule
        OK        & \( x < 0.8 \)                          & Constraint is comfortably fulfilled \\
        WARNING   & \( 0.8 \leq x \leq 1.0 \)             & Constraint is nearing violation     \\
        CRITICAL  & \( x > 1.0 \)                         & Constraint is violated              \\
        \bottomrule
    \end{tabular}
    \caption{Discretization of SLO values into discrete system states.}
    \label{tab:slo-states}
\end{table}


This ternary discretization introduces a minimal yet effective abstraction over the continuous SLO domain. It enables the Active Inference agent to reason over discrete system states when selecting actions. OK states reflect a healthy operating condition; WARNING states indicate a need for caution and CRITICAL states signal a SLO violation.


\subsection{Elasticity and Stream Parameter Control}
The \textit{Producer} serves as the central control entity for elasticity. It continuously monitors and adjusts three key \textit{quality parameters} of the video stream: (1) frames per second (\textit{fps}), (2) \textit{resolution}, and (3) \textit{inference quality}. While the producer directly sets fps and resolution by modulating task generation frequency and resizing video frames, inference quality refers to the YOLOv11 model variant employed by the worker nodes, e.g., \texttt{LOW} $\rightarrow$ \texttt{YOLOv11n}, \texttt{MEDIUM} $\rightarrow$ \texttt{YOLOv11s}, \texttt{HIGH} $\rightarrow$ \texttt{YOLOv11m}.

Although inference is executed on the workers, the producer dictates the inference quality of the entire system. To enforce a configuration change, it maintains a dedicated \textit{backlog} for each worker node. When the producer initiates a change (e.g., \texttt{MEDIUM} $\rightarrow$ \texttt{LOW}), it inserts the entry \texttt{CHANGE\_INFERENCE\_QUALITY=LOW} into the backlog of every registered worker. Upon the next task request, the worker's backlog is evaluated. If a configuration change is pending, the producer replies with a message of type \texttt{CHANGE} detailing all configuration changes the worker needs to make. This includes \texttt{CHANGE\_INFERENCE\_QUALITY=LOW}, instructing the worker to adapt its local inference model accordingly. Once the change is applied, the worker resumes normal task processing.

This design ensures that all nodes operate under a globally consistent inference configuration while minimizing coordination overhead.

\subsection{Active Inference Agent}
\label{sec:evaluation-implementation-active-infernce-agemt}
The control of the \textit{quality parameters} is delegated to the AIF agent running on the producer. Its objective is to uphold all defined Service Level Objectives (SLOs) while maximizing the Quality of Experience (QoE) through adaptive control of frame rate (FPS), resolution, and inference quality. The agent is implemented using pymdp \cite{heins_pymdp_2022} and operates continuously, executing one action-perception cycle every 500\,ms. Each cycle results in no action or a change in \textit{stream quality parameters}. 

\paragraph{Generative Model Construction.}
The agent operates a \textit{generative model} comprising the following components:
\begin{itemize}
  \item \textbf{Observation model \(P(o \mid s)\):} SLO values and quality parameters
  \item \textbf{Transition model \(P(s_{t+1} \mid s_t,a_t)\):} Latent configuration state
  \item \textbf{Prior preferences over observations \(P(o)\):} Strong preference for high quality parameters and equally strong aversion for SLO violations
  \item \textbf{Prior beliefs about hidden states \(P(s_0)\):} Inital quality parameter configuration.
\end{itemize}

\paragraph{Discrete State and Action Space Construction.}
Each configuration of the system is represented as a tuple of discrete quality parameter values:
\begin{itemize}
  \item \textbf{Resolution} $\in \{$480p, 720p, 1080p$\}$
  \item \textbf{Frame Rate (FPS)} $\in \{$10, 15, 20, 25, 30$\}$
  \item \textbf{Inference Quality} $\in \{$LOW, MEDIUM, HIGH$\}$
\end{itemize}

\paragraph{Actions}
The AIF agent operates using a \textit{relative control} strategy. Rather than selecting absolute parameter values, each action proposes an incremental change along one of the quality parameter dimensions. Specifically, the agent can:

\begin{itemize}
  \item \textbf{Frame Rate (FPS):} increase, decrease, or retain the current FPS setting.
  \item \textbf{Resolution:} increase, decrease, or retain the current resolution setting.
  \item \textbf{Inference Quality:} increase, decrease, or retain the YOLOv11 model quality.
\end{itemize}

A full action is a tuple $(a_\text{res}, a_\text{fps}, a_\text{qual})$, representing a joint change in the current configuration.


This results in an action space of $3 \times 3 \times 3 = 27$ discrete joint actions. Each action tuple $(\Delta_\text{res}, \Delta_\text{fps}, \Delta_\text{qual})$ specifies a directional adjustment in each dimension, where \(\Delta \in \{-1, 0, +1\}\). The current configuration is then updated accordingly, subject to parameter bounds. For instance, an action with \(\Delta_\text{fps} = -1\) reduces the frame rate by one level (e.g., from 30 to 25 FPS), unless the lower bound is already reached.

This relative formulation reduces the complexity of the policy space while enabling fine-grained and adaptive parameter control over time.

\paragraph{Online Model Learning.}
To adapt to non-stationary runtime conditions, the agent continuously refines its internal model during operation. This is achieved via the following update steps executed at each cycle:

\begin{itemize}
  \item \texttt{agent.update\_A(\(o_t\)):} Updates the observation model \( P(o \mid s) \) using the newly observed data.
  \item \texttt{agent.update\_B(\(Q(s_t)\)):} Updates the transition model \( P(s_{t+1} \mid s_t, a_t) \) based on the posterior belief \(Q(s_{t})\) from the previous timestep.
\end{itemize}

This learning mechanism allows the agent to gradually internalize the causal structure of the environment, e.g., how a reduction in resolution affects memory usage or how inference quality relates to task processing time. As a result, the agent becomes increasingly adept at selecting configurations that uphold SLOs under varying conditions.

\paragraph{Decision Prioritization and Trade-offs.}
The agent’s decision-making is shaped by its preference structure. Violations of any SLO are assigned a utility penalty large enough to outweigh the benefit of increasing QoE. This lexicographic structure ensures that:
\begin{itemize}
  \item SLO violations are strictly avoided whenever possible.
  \item QoE is maximized only within the bounds of safe operation.
\end{itemize}

In effect, the agent internalizes a dynamic control policy that minimizes long-term surprise~\cite{sedlak_adaptive_2024}, adapts to the environment via learning, and maintains operational stability under edge constraints.


\section{Worker Architecture}
\label{sec:implementation-worker-architecture}
Worker nodes form the computational core of the distributed pipeline. Their role is to execute inference tasks on data segments received from the producer and transmit the results to the collector. Each worker operates independently and adheres to a pull-based communication model, thereby ensuring a decentralized and scalable architecture.

Crucially, workers do not maintain internal logic for deciding whether a parameter change is required. Instead, they execute the received task exactly as instructed. If a message with type \texttt{CHANGE} is received, the worker applies all changes listed in the message's metadata—such as switching the YOLOv11 model or updating FPS/resolution targets—and then resumes regular processing. This design simplifies the worker logic and ensures synchronization across the system with minimal overhead.

\begin{figure}[htbp]
    \centering
    \includegraphics[width=\textwidth]{img/implementation/implementation_worker_architecture.drawio.pdf}
    \caption{Worker architecture. Detailing the process of requesting tasks, processing them, and sending the results to the collector}
    \label{fig:implementation-worker-architecture}
\end{figure}

\subsection{Startup and Registration}
Upon initialization, a worker registers with the producer using its unique identifier and receives the current stream configuration, including the active inference quality settings. The worker then initializes the YOLO models for future inference.

\subsection{3 Process Architecture}
To support concurrent operations and avoid resource contention, each worker is structured into three pipelines, each implemented as an independent process. This modular design enables clean separation of responsibilities and optimizes throughput by decoupling I/O-bound and compute-bound operations.

Most importantly, the use of three separate processes allows the worker to avoid bottlenecks caused by Python’s GIL and to exploit multicore CPUs effectively.

\subsection{Work Requesting Pipeline.}
This pipeline manages interaction with the producer and forwards received tasks for processing. It sends requests to the producer using a ZeroMQ REQ socket to request work. 

Each time the worker requests a new task, it attaches its most recent processing time to the request. This metric is used by the producer to compute both global and per-worker SLOs. By making performance reporting part of the task request cycle, the system avoids the need for separate monitoring channels and maintains low overhead.

Received tasks are buffered until they are ready to be handed over to the \textit{processing pipeline}. New work is requested whenever the task buffer is empty. If the producer facilitates a change in the worker's configuration and the worker receives a message with \texttt{type=CHANGE}, it immediately implements all changes detailed within the message. For example, switching YOLOv11 model.

\subsection{Processing Pipeline}
This process is dedicated to executing inference on received tasks. Upon receiving a task from the \textit{work requesting pipeline} with \texttt{type=INFERENCE}, it processes the tasks by executing inference using the currently active YOLOv11 model. The result is then re-packaged into a new task with \texttt{type=COLLECT} and forwarded to the \textit{result sending pipeline}.

\subsection{Result Sending Pipeline.}
This pipeline handles the transmission of processed results to the collector. It receives results from the \textit{processing pipeline} and enqueues them for transmission. Once ready, the results are then pushed to the collector.

\section{Collector Architecture}
The collector node forms the terminal stage of the distributed pipeline. Its primary responsibility is to receive results from worker nodes, reconstruct the output stream in the correct order, and optionally render or store the final output. 


\subsection{Result reordering}
The collector receives completed tasks from workers via a ZeroMQ PULL socket. Each task includes a unique \texttt{id} and a \texttt{stream\_key}, which together uniquely identify a specific frame within a logical stream. To maintain correct ordering, the collector uses a blocking dictionary that buffers out-of-order results until all preceding frames have arrived. If a task is delayed or lost—for example, due to a slow or faulty worker—it is dropped after a configurable timeout using a strike-based policy, thereby preserving the responsiveness of the system.

\subsection{Support for Concurrent Streams}
The architecture supports multiple concurrent data streams by tracking and assembling frames independently based on the \texttt{stream\_key} attribute. This is particularly important for the evaluation scenario involving changing workload, where multiple video streams are processed in parallel. Each stream is assigned its own logical ordering buffer, and the collector ensures that results from different streams do not interfere with one another.

The problem of stream reordering under asynchronous and heterogeneous compute conditions is well known in distributed systems. Apache Flink, for example, uses watermarks and checkpoint barriers to achieve consistent stream processing under variable network and compute latencies~\cite{carbone_apache_2015}. The collector in this prototype adopts a simplified design that reflects these core principles while remaining lightweight and suitable for resource-constrained edge deployments.
\chapter{Evaluation}
\label{chap:evaluation}

This chapter evaluates the methodology proposed in Chapter~\ref{chap:methodology} by applying the implementation described in Chapter~\ref{chap:implementation}. The objective is to empirically assess whether an Active Inference (AIF) agent can dynamically control computational elasticity to uphold Service Level Objectives (SLOs) while maximizing Quality of Experience (QoE) in edge computing environments. To provide a grounded comparison, we contrast the AIF agent with a simple heuristic agent under three controlled scenarios: a stable baseline, variable computational demand, and fluctuating computational budget. Each scenario simulates different real-world challenges encountered in distributed edge deployments, such as load bursts, resource contention, and partial node failure.

\section{Baseline: Heuristic Control Agent}
\label{sec:evaluation-heuristic}

To evaluate the effectiveness of the Active Inference approach, we implement a baseline agent that follows a fixed heuristic policy. This agent operates reactively based on SLO values observed during runtime. Whenever an SLO violation is detected, the heuristic agent adjusts stream parameters with the highest capacity. If the parameters have the same capacity, the decreasing action is taken according to a predefined order of importance: first reducing the inference quality, then the fps, and finally the resolution. When all SLos are overfullfilled, i.e., \(\text{SLO-Value} < 0.85\) and the current configuration is not maximal, it incrementally scales the quality parameters back up in the same order.
When the SLOs are fulfilled but too close to the critical threshold, i.e., \(0.85 \leq \text{SLO-Value} \leq 1\), no action is taken.

The heuristic agent does not use a generative model or perform any probabilistic inference. It cannot anticipate future states or reason under uncertainty. This makes it inherently limited to short-term, reactive control decisions. Despite its simplicity, this agent serves as a relevant baseline for highlighting the benefits of model-based adaptation under the Active Inference framework. The comparison focuses on control stability, responsiveness, and long-term QoE.

\section{Experimental Setup}
\label{sec:evaluation-setup}

\subsection{Environment}
\label{sec:evaluation-environment}

All experiments are conducted in a simulated edge computing environment using the prototype implementation described in Chapter~\ref{chap:implementation}. The system processes a live video stream of highway traffic using YOLOv11 inference for vehicle detection. The stream is segmented into frames by the producer and distributed to workers for parallel processing. The collector assembles the results into a final output stream.

Each worker simulates a bounded compute capacity by artificially delaying inference based on a configurable slowdown factor. The producer uses SLOs to track system state and controls three stream parameters: FPS, resolution, and inference quality. The experiments are run with both the AIF agent and the heuristic baseline to allow direct comparison.

Metrics are sampled continuously and include SLO values and Stream quality parameters (FPS, resolution, inference quality). Each experiment lasts 450 seconds.

\subsubsection{Hardware and Software}
All simulations are executed in a controlled single-machine environment to ensure reproducibility and eliminate variance introduced by distributed hardware. The system configuration is as follows:

\begin{table}[H]
\centering
\caption{Hardware and Software Configuration}
\label{tab:hardware-software}
\begin{tabular}{@{}ll@{}}
\toprule
\textbf{Component} & \textbf{Specification} \\
\midrule
Operating System & Windows 11, Version 23H2 (Build 22631.5335) \\
Python Runtime & Python 3.12.2 \\
CPU & AMD Ryzen 7 7800X3D (8 cores, 16 threads) \\
GPU & Nvidia GeForce GTX 1660Ti (MSI GTX Ti Ventus XS OC) \\
GPU Driver Version & 576.88 \\
Installed CUDA Version & 12.9 \\
CUDA Compiler Version & 12.6 (cuda\_12.6.r12.6/compiler.34841621\_0) \\
Memory & 32\,GB DDR5 RAM @ 4800\,MHz (dual channel) \\
Storage & WD Black SN770 2\,TB NVMe SSD \\
\bottomrule
\end{tabular}
\end{table}


This configuration provides sufficient headroom for parallel task execution and GPU-accelerated inference, while reflecting performance characteristics common in high-end edge servers and developer workstations.

\subsection{Scenario A: Base Case}
\label{sec:evaluation-base}

This scenario evaluates agents' behavior in a stable and controlled environment. It serves as the baseline to assess the steady-state behavior of both agents. The goal is to analyze convergence behavior, control stability, and SLO compliance in the absence of external perturbations.

The experiment is conducted using three worker nodes, each assigned a fixed computational capacity of 60\%, 50\%, and 40\%, respectively

These normalized capacity values simulate processing slowdowns by artificially extending inference time. The capacities are deliberately unequal to reflect the heterogeneity typical of real-world edge environments, where nodes differ in hardware capabilities, energy budgets, or concurrent workloads.

A single video stream is processed at a constant source frame rate. At the beginning of the experiment, the system is initialized with the highest possible configuration: 30 FPS, 1080p resolution, and the YOLOv11m model. This starting point is intentionally unsustainable under the given compute constraints and is expected to trigger elastic adaptation by the controller. The goal is to evaluate whether the AIF agent reaches optimal QoE faster, switches less frequently, and maintains fewer SLO violations compared to the heuristic agent.

\subsection{Scenario B: Variable Computational Demand}
\label{sec:evaluation-variable-demand}

This scenario simulates a dynamic workload by varying the number of concurrent video streams during runtime. Three workers are assigned fixed computational capacities of 80\%, 75\%, and 70\%, respectively. The producer initiates multiple streams during a certain stream multiplier schedule, at key points during the runtime:

\begin{itemize}
    \item 0\%–-25\%: 1 stream
    \item 25\%-–50\%: 3 streams
    \item 50\%–-75\%: 2 streams
    \item 75\%–-100\%: 1 stream
\end{itemize}

Each stream is identified by a unique \texttt{stream\_key} and processed independently through the same distributed pipeline. The worker nodes receive tasks from all streams interleaved and must process them within their capacity constraints.

The purpose of this scenario is to evaluate how quickly and effectively the AIF and heuristic agents adapt to abrupt changes in computational demand. It also assesses the system’s ability to reconfigure quality parameters across multiple streams in parallel without violating SLOs.

\subsection{Scenario C: Variable Computation Budget}
\label{sec:evaluation-variable-budget}

This scenario evaluates controller robustness under fluctuating compute resources. The system starts with two active workers, each with 50\% compute capacity. At 25\% of the runtime, one worker is deactivated, effectively halving the available compute budget. At 75\%, the worker is reactivated, returning to the original resource budget.

This emulates a temporary node failure. The producer continues to generate tasks during the outage period, and the controller must react by reducing quality parameters to avoid SLO violations.

This scenario evaluates the ability of the AIF agent to learn the new transition dynamics during the temporary failure and to revert to a higher-quality configuration once resources are restored. The heuristic controller, by contrast, reacts only to observable SLO violations and does not model temporal dependencies.

\section{Results}

This section presents and interprets the results of the evaluation comparing an Active Inference (AIF) based agent with a baseline heuristic controller across three simulation scenarios. The analysis focuses on five core metrics that quantify SLO compliance, system stability, and output stream quality. Results are presented comparatively and are followed by a summary table aggregating the main findings.

\subsection{Evaluation Metrics and Their Significance}

The evaluation is based on five key metrics:

\begin{itemize}
  \item \textbf{Percentage Time All SLOs Fulfilled Simultaneously}: Measures how often all SLO constraints were met concurrently. Higher values indicate strong global constraint satisfaction.
  \item \textbf{Average SLO Fulfillment Rate}: Captures the average percentage of SLOs satisfied at each timestep. This is less strict and captures partial fulfillment.
  \item \textbf{Average Overall Stream Quality Score}: Quantifies the average quality of the stream (resolution, FPS, YOLO model quality). Higher scores imply better Quality of Experience (QoE).
  \item \textbf{Maximum Consecutive Timesteps with SLO Violations}: Indicates the worst-case duration of persistent SLO violations. Lower values reflect faster adaptation.
  \item \textbf{Average Timesteps to Reach Stable Quality Configuration}: Captures how quickly the agent converges to a stable parameter configuration. Lower values reflect faster convergence.
\end{itemize}

An overview of the metrics is provided in table~\ref{tab:results_summary}.

\begin{table}[h!]
\centering
\caption{Summary of Core Evaluation Metrics}
\label{tab:results_summary}
\begin{tabular}{@{}lllll@{}}
\toprule
\textbf{Scenario} & \textbf{Metric} & \textbf{AIF Agent} & \textbf{Heuristic} & \textbf{$\Delta$} \\
\midrule
\multirow{5}{*}{Base} 
& \% Time All SLOs Met & 0.982 & 0.9559 & +2.73\% \\
& Avg. SLO Fulfillment & 0.9938 & 0.9871 & +0.67\% \\
& Stream Quality Score & 0.8007 & 0.7577 & +5.68\% \\
& Max SLO Violation Streak & 16 & 14 & +14.29\% \\
& Stable Config Time & 1.0 & 0.875 & +14.29\% \\
\midrule
\multirow{5}{*}{Budget} 
& \% Time All SLOs Met & 0.9615 & 0.8843 & +7.72\% \\
& Avg. SLO Fulfillment & 0.9827 & 0.9675 & +1.52\% \\
& Stream Quality Score & 0.6077 & 0.7541 & --19.40\% \\
& Max SLO Violation Streak & 31 & 18 & +72.22\% \\
& Stable Config Time & 1.0 & 0.9412 & +6.24\% \\
\midrule
\multirow{5}{*}{Demand} 
& \% Time All SLOs Met & 0.7126 & 0.8518 & --13.92\% \\
& Avg. SLO Fulfillment & 0.9073 & 0.9488 & --4.15\% \\
& Stream Quality Score & 0.7663 & 0.8216 & --6.73\% \\
& Max SLO Violation Streak & 259 & 68 & +280.88\% \\
& Stable Config Time & 1.0 & 0.9697 & +3.12\% \\
\bottomrule
\end{tabular}
\end{table}



\subsection{Scenario A: Base Case}

In the base case scenario, the AIF agent outperformed the heuristic agent on all relevant SLO metrics. It achieved a \textbf{full SLO satisfaction} of \(98.2\%\), compared to \(95.59\%\) for the heuristic---a relative improvement of \(+2.73\%\). Similarly, the \textbf{average SLO fulfillment rate} increased from \(98.71\%\) to \(99.38\%\), and the \textbf{average overall stream quality score} improved from \(0.7577\) to \(0.8007\), a relative increase of \(+5.68\%\).

Despite slightly longer maximum violation durations (AIF: 16 timesteps, Heuristic: 14), a relative increase of \(+14.29\%\), the AIF agent demonstrated better convergence behavior, reaching a stable configuration in just \(1 \text{timestep}\) on average, which is \(+14.29\%\) longer than the heuristic baseline. This supports the observation that the AIF agent front-loads exploratory adaptation and quickly stabilizes its belief about optimal stream parameters.

The improved performance is due to the agent’s epistemic behavior: once a satisfactory configuration is learned, the agent avoids further changes unless necessary. The heuristic, by contrast, continues to greedily exploit temporary resource availability, leading to greater parameter fluctuation and less consistent global performance.

\subsection{Scenario B: Variable Computational Demand}

This scenario presented the most challenging environment: the computational demand increased significantly as the number of concurrent streams tripled. Here, the heuristic agent achieved better SLO metrics, with \(85.18\%\) simultaneous SLO satisfaction versus \(71.26\%\) for the AIF agent---a \(-13.92\%\) difference.

The \textbf{average SLO fulfillment rate} followed a similar pattern: \(94.88\%\) for the heuristic versus \(90.73\%\) for AIF. However, this apparent advantage comes at a cost. The \textbf{maximum SLO violation duration} for the AIF agent was \(259\text{timesteps}\), which is \(+280.88\%\) higher than the heuristic’s \(68\text{timesteps}\), revealing poor responsiveness to abrupt environmental change.

Despite the inferior SLO metrics, the AIF agent maintained a relatively high \textbf{stream quality score} of \(0.7663\), which is only \(-6.73\%\) lower than the heuristic’s \(0.8216\). This highlights that the AIF agent, once it begins adaptation, avoids unnecessary quality degradation and maintains equilibrium once learned.

\subsection{Scenario C: Variable Computational Budget}

In this more dynamic setting, where the available computational resources change over time, the AIF agent maintained high performance. It achieved \(\textbf{96.15}\%\) simultaneous SLO satisfaction compared to \(88.43\%\) for the heuristic---a \(+7.72\%\) advantage. The \textbf{average SLO fulfillment rate} also improved from \(96.75\%\) to \(98.27\%\).

However, the cost of the AIF agent’s equilibrium-seeking behavior is visible in the \textbf{maximum consecutive SLO violation streak}, which rose from \(18\) (heuristic) to \(31\) (AIF), a \(+72.22\%\) increase. This indicates a slower recovery from extreme events where resource availability drops sharply. Nevertheless, once recovery occurs, the AIF agent stabilizes more effectively and avoids repeated violations.

The \textbf{stream quality score} for the AIF agent decreased to \(0.6077\), compared to \(0.7541\) for the heuristic, a drop of \(-19.40\%\). This is a direct consequence of the AIF agent favoring SLO compliance over QoE maximization when resource constraints are tight.

\subsection{Temporal Adaptation Analysis}

Temporal inspection of the stream quality parameters reveals that the AIF agent performs a concentrated burst of exploratory changes early in the simulation. After this initial phase, its configuration changes become infrequent and deliberate. This confirms that the AIF agent seeks an internal equilibrium, minimizing surprise through conservative parameter updates once it has developed high confidence in its generative model.

In contrast, the heuristic controller shows a reactive pattern throughout the simulation: FPS, resolution, and inference quality fluctuate frequently in response to transient load changes. While this can lead to higher temporary QoE, it also increases the risk of violating SLOs due to overutilization.

\subsection{Discussion of Agent Strategies}

The AIF agent’s superior performance in stable and moderately dynamic environments reflects its core design principles. By minimizing expected free energy, it balances the dual goals of satisfying constraints (pragmatic value) and reducing uncertainty (epistemic value). This leads to early adaptation followed by long-term consistency.

The heuristic controller lacks this capability for uncertainty modeling and acts purely on observed feedback. While this allows for aggressive optimization under predictable load, it fails under volatile or ambiguous conditions.

In highly dynamic environments, particularly with rapid changes in computational demand, the AIF agent’s preference for prior beliefs results in sluggish adaptation. Without online model updates, the agent fails to adapt quickly to new workload regimes, highlighting a key area for future improvement.

\chapter{Discussion}
\section{AIF on Worker, worker sends preferences to Producer}
AIF auch auf dem Worker. Worker misst die Qualität seines outputs(frames). 
Der Worker teilt seine preference für inference quality dem Producer mit. 
Der Producer übernimmt die joint preferences von den workerm

Welches problem Löst das? Beispeil:
Worker läuft mit inference quality  5/10. Er hätte die ressourcen seine inference quality auf 6/10 zu erhöhen.
Jedoch weis er durch kombination von momentanen messergebnissen und seinen vorherigen states (encoded im generative model), dass eine erhöhun gauf 6/10 infernce quality die qualität des outputs nicht verbessert und nur ressourcen verschwenden würde.

\section{very slow worker nodes}
The load balancing design pattern maximizes the resource usage in heterogeneous workers. But it doesnt resolve the issue of extremely slow worker nodes. When a worker is below a certain threshold for processing a single \textbf{task}, then this tasks may not be used in the resulting output. This is because by the time the slow worker is done processing its tasks, the output-stream has already moved past this point. Meaning that the work done by the node is essential useless.

For example:
If we have video-stream running at 30fps, this means that there has to be a new frame every \(33\text{ms}\). If the delay between the source-stream and the output-stream is \(1\text{s}\), then a worker has at most \(1\text{s}\) for processing a single task. If the worker is unable to processes the requested task within this time frame, then the result-task will discarded by the collector as it is ''behind'' the current stream, i.e., the most recent task that has reached output has a higher id then the task that was submitted by the slow worker node. This results in lost information, e.g, dropped frames for a video stream and the computational power of the slow worker node is wasted.

To combat this, we implemented the SLO \textit{Average Task Processing Time per worker}, meaning if a single worker is too slow, then the stream parameter configuration is changed. This comes with the downside that when a single worker is way slower than the other workers, it might drag down the QoE of the entire system.
\chapter{Conclusion}
\label{chap:conclusion}

This thesis investigated the application of the Active Inference (AIF) framework for upholding Service Level Objectives (SLOs) in distributed edge computing scenarios. A self-adaptive stream processing pipeline was developed, leveraging AIF to regulate elasticity along three dimensions: frame rate, resolution, and model quality. The objective was to maintain high Quality of Experience (QoE) while operating under resource constraints.

The implemented prototype integrates an AIF agent at the producer node. By minimizing expected free energy, the agent continuously selects actions that reduce the risk of future SLO violations. Online learning through parameter updates enables the agent to adapt its generative model during runtime, facilitating behavior adjustment without requiring offline retraining.

Evaluation showed that the AIF-based elasticity mechanism achieves comparable or superior global SLO compliance relative to a heuristic baseline across varying conditions.While the heuristic performs better in maximizing stream quality under high load, the AIF agent consistently prioritizes SLO fulfillment, even at the cost of reduced output quality.

The Active Inference agent demonstrates clear advantages in stable and moderately dynamic environments, where it maintains high SLO compliance through deliberate and consistent adaptation. Its preference for equilibrium leads to minimal parameter oscillation and robust long-term performance. However, in highly volatile scenarios with abrupt workload changes, the agent exhibits delayed adaptation due to its reliance on prior beliefs. This highlights a key limitation of the current approach: a lack of fast re-adaptation mechanisms.

While the evaluation focused on video inference, the presented approach generalizes to other data modalities, such as LIDAR, audio streams, and sensor telemetry. 

In summary, this thesis contributes a principled and extensible method for adaptive control in edge computing. By unifying probabilistic reasoning, preference-driven adaptation, and online learning, the proposed system provides a novel foundation for resilient, self-organizing stream processing under uncertainty.



% Remove following line for the final thesis.
%\input{intro.tex} % A short introduction to LaTeX.

\backmatter

% Declare the use of AI tools as mentioned in the statement of originality.
% Use either the English aitools or the German kitools.
\begin{aitools}

In the course of writing this thesis, several generative AI-based tools were used to support specific tasks related to grammar correction, LaTeX troubleshooting, and literature note-taking. The tools listed below were used selectively and within the scope permitted for academic writing support.

\begin{itemize}
    \item \textbf{Grammarly} -- Used primarily for spell and grammar checking of written text, limited to basic proofreading. Available at \url{https://app.grammarly.com/}.
    
    \item \textbf{Writefull (Overleaf Integration)} -- Integrated into Overleaf to provide in-editor grammar and language feedback. Used solely to identify minor writing issues. More information at \url{https://www.writefull.com/}.
    
    \item \textbf{Overleaf AI Assist} -- Employed to resolve LaTeX syntax and compilation errors during document preparation. Details can be found at \url{https://www.overleaf.com/learn/how-to/AI_Assist}.
    
    \item \textbf{Google NotebookLM} -- Used as a digital notebook for organizing literature notes and retrieving specific excerpts from uploaded documents. Accessible at \url{https://notebooklm.google.com/}.
\end{itemize}

All generative AI tools were used transparently, with careful attention to maintaining the integrity and originality of the thesis content.
\end{aitools}

% Use an optional list of figures.
\listoffigures % Starred version, i.e., \listoffigures*, removes the toc entry.

% Use an optional list of tables.
\cleardoublepage % Start list of tables on the next empty right hand page.
\listoftables % Starred version, i.e., \listoftables*, removes the toc entry.

% Use an optional list of alogrithms.
\listofalgorithms
\addcontentsline{toc}{chapter}{List of Algorithms}

% Add an index.
\printindex

% Add a glossary.
\printglossaries

% Add a bibliography.
\bibliographystyle{ieeetr}
\bibliography{bib}

\appendix
\appendix
\chapter{Simulation Results}
´\label{appendix:chap:simulation-results}

This appendix provides the detailed results for all evaluation scenarios. For each simulation, we present:
\begin{itemize}
    \item Key performance metrics in tabular form
    \item Visualizations of stream quality, SLO values, and task distribution
\end{itemize}


\section{Scenario A: Base Case}

\begin{table}[h!]
\centering
\small
\caption{Complete Comparison of All Evaluation Metrics - Base Scenario}
\label{tab:complete_results_basic}
\begin{tabular}{@{}llll@{}}
\toprule
\textbf{Metric} & \textbf{AIF} & \textbf{Heuristic} & \textbf{$\Delta$} \\
\midrule
Avg. Global Processing Time SLO Value & 0.9088 & 0.9424 & -3.56\% \\
Avg. Worker Processing Time SLO Value & 0.2961 & 0.3013 & -1.72\% \\
Avg. Memory Usage SLO Value & 0.5827 & 0.5814 & +0.23\% \\
Avg. Stream Quality Score & 0.800 & 0.739 & +8.30\% \\
Avg. Queue Size SLO Value & 0.0061 & 0.0083 & -26.46\% \\
Avg. SLO Fulfillment Rate & 0.994 & 0.995 & -0.08\% \\
Avg. SLO Value Across All SLOs & 0.4484 & 0.4583 & -2.16\% \\
Avg. Time to Stable Config & 1.0 & 0.8 & +33.33\% \\
Global Processing Time Violation Severity & 2.02 & 1.47 & +37.33\% \\
Global Processing Time Stability CoV & 0.6094 & 0.3252 & +87.39\% \\
Worker Processing Time Violation Severity & 3.68 & 2.05 & +79.43\% \\
Worker Processing Time Stability CoV & 1.3924 & 0.8580 & +62.28\% \\
Global Processing Time SLO Fulfillment Rate & 0.983 & 0.986 & -0.28\% \\
Max Consecutive SLO Violations & 15 & 10 & +50.00\% \\
Memory Usage Violation Severity & 0.00 & 0.00 & +0.00\% \\
Memory Usage SLO Fulfillment Rate & 1.000 & 1.000 & +0.00\% \\
Memory Usage Stability CoV & 0.0119 & 0.0065 & +82.28\% \\
Queue Size Violation Severity & 0.00 & 0.00 & +0.00\% \\
Queue Size SLO Fulfillment Rate & 1.000 & 1.000 & +0.00\% \\
Queue Size Stability CoV & 10.6678 & 9.3604 & +13.97\% \\
Time All SLOs Met & 0.983 & 0.986 & -0.28\% \\
Worker Processing Time SLO Fulfillment Rate & 0.993 & 0.993 & -0.02\% \\
\bottomrule
\end{tabular}
\end{table}


\begin{figure}[h]
    \centering
    \includegraphics[width=\textwidth]{img/results/basic/active_inference_relative_control_quality_metrics.pdf}
    \caption{AIF –- Quality Metrics Over Time (Base Case)}
\end{figure}
\begin{figure}[h]
    \centering
    \includegraphics[width=\textwidth]{img/results/basic/active_inference_relative_control_quality_metrics.pdf}
    \caption{AIF –- Quality Metrics Over Time (Base Case)}
\end{figure}
\begin{figure}[h]
    \centering
    \includegraphics[width=\textwidth]{img/results/basic/active_inference_relative_control_slo_values.pdf}
    \caption{AIF –- SLO Metrics Over Time (Base Case)}
\end{figure}
\begin{figure}[h]
    \centering
    \includegraphics[width=0.5\textwidth]{img/results/basic/active_inference_relative_control_task_distribution_pie.pdf}
    \caption{AIF –- Task Distribution (Base Case)}
\end{figure}


\begin{figure}[h]
    \centering
    \includegraphics[width=\textwidth]{img/results/basic/heuristic_quality_metrics.pdf}
    \caption{Heuristic –- Quality Metrics Over Time (Base Case)}
\end{figure}
\begin{figure}[h]
    \centering
    \includegraphics[width=\textwidth]{img/results/basic/heuristic_slo_values.pdf}
    \caption{Heuristic –- SLO Metrics Over Time (Base Case)}
\end{figure}
\begin{figure}[h]
    \centering
    \includegraphics[width=0.5\textwidth]{img/results/basic/heuristic_task_distribution_pie.pdf}
    \caption{Heuristic –- Task Distribution (Base Case)}
\end{figure}

\clearpage
\section{Scenario B: Variable Computational Demand}

\begin{table}[h!]
\centering
\small
\caption{Complete Comparison of All Evaluation Metrics - Demand Scenario}
\label{tab:complete_results_variable_computational_demand}
\begin{tabular}{@{}llll@{}}
\toprule
\textbf{Metric} & \textbf{AIF} & \textbf{Heuristic} & \textbf{$\Delta$} \\
\midrule
Avg. Global Processing Time SLO Value & 0.7157 & 0.8400 & -14.80\% \\
Avg. Worker Processing Time SLO Value & 0.1991 & 0.2275 & -12.49\% \\
Avg. Memory Usage SLO Value & 0.5944 & 0.5758 & +3.22\% \\
Avg. Stream Quality Score & 0.657 & 0.813 & -19.23\% \\
Avg. Queue Size SLO Value & 0.1702 & 0.0704 & +141.80\% \\
Avg. SLO Fulfillment Rate & 0.960 & 0.971 & -1.07\% \\
Avg. SLO Value Across All SLOs & 0.4198 & 0.4284 & -2.01\% \\
Avg. Time to Stable Config & 1.0 & 1.0 & +3.85\% \\
Global Processing Time Violation Severity & 0.90 & 0.57 & +56.10\% \\
Global Processing Time Stability CoV & 0.7259 & 0.3924 & +85.00\% \\
Worker Processing Time Violation Severity & 1.32 & 0.99 & +34.25\% \\
Worker Processing Time Stability CoV & 0.9846 & 0.4946 & +99.08\% \\
Global Processing Time SLO Fulfillment Rate & 0.921 & 0.914 & +0.78\% \\
Max Consecutive SLO Violations & 77 & 28 & +175.00\% \\
Memory Usage Violation Severity & 0.00 & 0.00 & +0.00\% \\
Memory Usage SLO Fulfillment Rate & 1.000 & 1.000 & +0.00\% \\
Memory Usage Stability CoV & 0.0158 & 0.0081 & +95.26\% \\
Queue Size Violation Severity & 1.14 & 0.30 & +274.88\% \\
Queue Size SLO Fulfillment Rate & 0.925 & 0.971 & -4.64\% \\
Queue Size Stability CoV & 3.4008 & 3.5852 & -5.14\% \\
Time All SLOs Met & 0.906 & 0.902 & +0.49\% \\
Worker Processing Time SLO Fulfillment Rate & 0.994 & 0.998 & -0.37\% \\
\bottomrule
\end{tabular}
\end{table}



\begin{figure}[h]
    \centering
    \includegraphics[width=\textwidth]{img/results/variable_computational_demand/active_inference_relative_control_quality_metrics.pdf}
    \caption{AIF –- Quality Metrics Over Time (Demand)}
\end{figure}
\begin{figure}[h]
    \centering
    \includegraphics[width=\textwidth]{img/results/variable_computational_demand/active_inference_relative_control_slo_values.pdf}
    \caption{AIF –- SLO Metrics Over Time (Demand)}
\end{figure}
\begin{figure}[h]
    \centering
    \includegraphics[width=0.5\textwidth]{img/results/variable_computational_demand/active_inference_relative_control_task_distribution_pie.pdf}
    \caption{AIF –- Task Distribution (Demand)}
\end{figure}



\begin{figure}[h]
    \centering
    \includegraphics[width=\textwidth]{img/results/variable_computational_demand/heuristic_quality_metrics.pdf}
    \caption{Heuristic –- Quality Metrics Over Time (Demand)}
\end{figure}
\begin{figure}[h]
    \centering
    \includegraphics[width=\textwidth]{img/results/variable_computational_demand/heuristic_slo_values.pdf}
    \caption{Heuristic –- SLO Metrics Over Time (Demand)}
\end{figure}
\begin{figure}[h]
    \centering
    \includegraphics[width=0.5\textwidth]{img/results/variable_computational_demand/heuristic_task_distribution_pie.pdf}
    \caption{Heuristic –- Task Distribution (Demand)}
\end{figure}

\clearpage
\section{Scenario C: Variable Computational Budget}

\begin{table}[h!]
\centering
\small
\caption{Complete Comparison of All Evaluation Metrics - Budget Scenario}
\label{tab:complete_results_variable_computational_budget}
\begin{tabular}{@{}llll@{}}
\toprule
\textbf{Metric} & \textbf{AIF} & \textbf{Heuristic} & \textbf{$\Delta$} \\
\midrule
Avg. Global Processing Time SLO Value & 0.8698 & 0.9325 & -6.72\% \\
Avg. Worker Processing Time SLO Value & 0.2258 & 0.2369 & -4.68\% \\
Avg. Memory Usage SLO Value & 0.5213 & 0.5190 & +0.44\% \\
Avg. Stream Quality Score & 0.672 & 0.754 & -10.90\% \\
Avg. Queue Size SLO Value & 0.0328 & 0.0445 & -26.26\% \\
Avg. SLO Fulfillment Rate & 0.987 & 0.985 & +0.26\% \\
Avg. SLO Value Across All SLOs & 0.4124 & 0.4332 & -4.80\% \\
Avg. Time to Stable Config & 1.0 & 0.9 & +11.11\% \\
Global Processing Time Violation Severity & 1.51 & 0.66 & +129.91\% \\
Global Processing Time Stability CoV & 0.7728 & 0.4955 & +55.96\% \\
Worker Processing Time Violation Severity & 2.19 & 0.80 & +175.71\% \\
Worker Processing Time Stability CoV & 1.0511 & 0.5502 & +91.06\% \\
Global Processing Time SLO Fulfillment Rate & 0.967 & 0.954 & +1.37\% \\
Max Consecutive SLO Violations & 23 & 16 & +43.75\% \\
Memory Usage Violation Severity & 0.00 & 0.00 & +0.00\% \\
Memory Usage SLO Fulfillment Rate & 1.000 & 1.000 & +0.00\% \\
Memory Usage Stability CoV & 0.0052 & 0.0132 & -61.07\% \\
Queue Size Violation Severity & 0.10 & 0.16 & -41.14\% \\
Queue Size SLO Fulfillment Rate & 0.987 & 0.991 & -0.38\% \\
Queue Size Stability CoV & 5.1770 & 3.2498 & +59.30\% \\
Time All SLOs Met & 0.967 & 0.954 & +1.37\% \\
Worker Processing Time SLO Fulfillment Rate & 0.995 & 0.995 & +0.09\% \\
\bottomrule
\end{tabular}
\end{table}



\begin{figure}[h]
    \centering
    \includegraphics[width=\textwidth]{img/results/variable_computational_budget/active_inference_relative_control_quality_metrics.pdf}
    \caption{AIF –- Quality Metrics Over Time (Budget)}
\end{figure}
\begin{figure}[h]
    \centering
    \includegraphics[width=\textwidth]{img/results/variable_computational_budget/active_inference_relative_control_slo_values.pdf}
    \caption{AIF –- SLO Metrics Over Time (Budget)}
\end{figure}
\begin{figure}[h]
    \centering
    \includegraphics[width=0.5\textwidth]{img/results/variable_computational_budget/active_inference_relative_control_task_distribution_pie.pdf}
    \caption{AIF –- Task Distribution (Budget)}
\end{figure}


\begin{figure}[h]
    \centering
    \includegraphics[width=\textwidth]{img/results/variable_computational_budget/heuristic_quality_metrics.pdf}
    \caption{Heuristic –- Quality Metrics Over Time (Budget)}
\end{figure}
\begin{figure}[h]
    \centering
    \includegraphics[width=\textwidth]{img/results/variable_computational_budget/heuristic_slo_values.pdf}
    \caption{Heuristic –- SLO Metrics Over Time (Budget)}
\end{figure}
\begin{figure}[h]
    \centering
    \includegraphics[width=0.5\textwidth]{img/results/variable_computational_budget/heuristic_task_distribution_pie.pdf}
    \caption{Heuristic –- Task Distribution (Budget)}
\end{figure}


\end{document}